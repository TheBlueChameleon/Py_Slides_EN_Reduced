% =========================================================================== %

\begin{frame}[t,plain]
\titlepage
\end{frame}

% =========================================================================== %

\begin{frame}{Recap}
%
\begin{columns}[T]
\column{.5\linewidth}
\begin{itemize}
\item Modules
	\begin{itemize}
	\item Load with \inPy{import modulename}
	\item Use objects with \inPy{modulename.symbol}
	\end{itemize}
\item \inPy{list}s
	\begin{itemize}
	\item Data containers
	\item Elements in [brackets, separated by commas]
	\item Access with [index in brackets]
	\item Negative indices: count from end, backwards
	\end{itemize}
\end{itemize}
%
\column{.5\linewidth}
\begin{itemize}
\item Slices
	\begin{itemize}
	\item \emph{Extended index}
	\item Creates new (temporary) \inPy{list}
	\item \hspace{0pt} [start : stop : stride]
	\item any of these parameters can be omitted
	\end{itemize}
\item References and copies
	\begin{itemize}
	\item Assignment via \texttt{=} actually assigns references
	\item Module copy
	\item \inPy{copy.deepcopy} for sublists
	\end{itemize}
\item mutable / immutable objects
\end{itemize}

\end{columns}
%
\begin{center}
	\emph{Any Questions?}
\end{center}
%
\end{frame}

% =========================================================================== %

\begin{frame}[fragile]{Chapter 4}
%
\begin{itemize}
\item \inPy{str}ings
\item \inPy{tuple}s
\item \inPy{set}s and \inPy{frozenset}s
\item \inPy{ranges}
\item \inPy{dict}ionaries
\item Special functions and operators for data containers
\end{itemize}
%
\end{frame}

% =========================================================================== %

\begin{frame}[fragile]{Strings}
%
\begin{columns}[T]
\column{.5\linewidth}
\begin{itemize}
\item We already know: texts in \inPy{'single'} or \inPy{"double"} quotes
\item Internally: \emph{immutable list of characters}
\item Known techniques applicable:
	\begin{itemize}
	\item Read access via indices -- pick individual characters
	\item Slicing -- pick substring
	\end{itemize}
\item But: \emph{immutable}
	\begin{itemize}
	\item Only string as a whole can be constructed
	\item Not possible to exchange single characters
	\end{itemize}
\end{itemize}
%
\column{.5\linewidth}
\begin{codebox}[Example: Strings]
\begin{minted}[linenos, fontsize=\scriptsize]{python}
text = "These go to eleven"

print(text[0])     # 'T'
print(text[-1])    # 'n'
print(text[:5])    # 'These'

# text[0] = "t"    -- no write access

text = "It's one louder" 
# newly constructed string: okay
\end{minted}
\end{codebox}
\end{columns}
%
\end{frame}

% =========================================================================== %

\begin{frame}[fragile]{\inPy{tuple}s}
%
\begin{columns}[T]
\column{.5\linewidth}
\begin{itemize}
\item Like \inPy{list}s: Data container
\item But: \emph{immutable}
\item Syntax: in (parenthesis)
\item No \texttt{append}, \texttt{delete}, \texttt{sort}, ...
\item Addition constructs a new \inPy{tuple}
\item Often: Return value of functions
	\begin{itemize}
	\item Example \inPy{divmod}: quotient and remainder as \inPy{tuple}
	\item \inPy{divmod(20, 3)} \thus ~ \inPy{(6, 2)}
	\end{itemize}
\item Unpacking: \inPy{x, y = (1, 2)}
\end{itemize}
%
\column{.5\linewidth}
\begin{codebox}[Example: \texttt{tuple}s]
\begin{minted}[linenos, fontsize=\scriptsize]{python}
t = (1, 2, 3)
print(t)        #  (1, 2, 3)
print(t[0])     #  1
# t[0] = -1     -- no write access
t = (-1, -2, -3)

t = tuple([4, 5, 6])
# type conversion list -> tuple

t = tuple("text")
# type conversion string -> tuple
# ('t', 'e', 'x', 't')
\end{minted}
\end{codebox}
\end{columns}
%
\end{frame}

% =========================================================================== %

\begin{frame}[fragile, t]{\inPy{set}s and \inPy{frozenset}s}
%
\vspace{-15pt}
\begin{columns}[t]
\column{.5\linewidth}
\begin{itemize}
\item \inPy{set}s
	\begin{itemize}
	\item Data containers for \emph{unique values}
	\item \emph{mutable}
	\item Syntax: in \{curly brackets\}
	\item Important methods: \texttt{add}, \texttt{clear}, \texttt{difference}, \texttt{intersection}, \texttt{union}
	\item Addition is not defined for \inPy{set}s (but logical or (\texttt{|}), as well as logical and (\texttt{\&}))
	\end{itemize}
\item \inPy{frozenset}s
	\begin{itemize}
	\item Just like \inPy{set}s, but \emph{immutable}
	\item No syntax element of their own, but via constructor: \inPy{variable = frozenset(container)}
	\end{itemize}
\end{itemize}
%
\column{.5\linewidth}
\begin{codebox}[Example: \texttt{set}s]
\begin{minted}[linenos, fontsize=\scriptsize]{python}
s = {1, 2, 3, 2}
print(s)        #  {1, 2, 3}
# s[0] = -1     
# no write access to individual elements

s.add(4)        # but updates possible
s = s.intersection({2, 4, 6})
print(s)        # {2, 4}

s = set("aardvark")
print(s)        #{'r', 'v', 'k', 'd', 'a'}
# order is not preserved
\end{minted}
\end{codebox}
\end{columns}
%
\end{frame}

% =========================================================================== %

\begin{frame}[fragile]{\inPy{range}s}
%
\vspace{-15pt}
\begin{columns}[t]
\column{.5\linewidth}
\begin{itemize}
\item Idea: low-memory-container for \inPy{int}egers
\item \inPy{start, stop, stride}
\item Really only stores these three values
\item Generates everything in between \enquote{on demand}
\item \texttt{start} inclusive, \texttt{stop} exclusive
\item \texttt{stride}: optional, default: \inPy{1}
\item \texttt{stride} may be negative
\item \texttt{start}, \texttt{stop}, \texttt{stride}: \emph{\inPy{int}egers only}
\end{itemize}
%
\column{.5\linewidth}
\begin{codebox}[Example: \texttt{range}s]
\begin{minted}[linenos, fontsize=\scriptsize]{python}
r = range(1, 10, 2)
print(r)         # range(1, 10, 2)
print(list(r))   # [1, 3, 5, 7, 9]

r = range(1, 5)
print(list(r))   # [1, 2, 3, 4]

print(list(range(5, 0, -1)))
# [5, 4, 3, 2, 1]

\end{minted}
\end{codebox}
\end{columns}
%
\end{frame}

% =========================================================================== %

\begin{frame}[fragile]{\inPy{dict}ionaries}
%
\vspace{-15pt}
\begin{columns}[t]
\column{.5\linewidth}
\begin{itemize}
\item \enquote{\inPy{list} with arbitrary \emph{keys} as indices}
\item key: (almost) arbitrary value, \eg a string
\item \{curly brackets\}, key-value pairs, linked with a colon (:)
\item Important Methods:
	\begin{itemize}
	\item \texttt{keys} -- returns (something like) a \inPy{list} of keys
	\item \texttt{values} -- returns (something like) a \inPy{list} of values
	\item \texttt{items} -- returns (something like) a \inPy{list} of \inPy{tuple}s (key, value)
	\end{itemize}
\end{itemize}
%
\column{.5\linewidth}
\begin{codebox}[Example: \texttt{range}s]
\begin{minted}[linenos, fontsize=\scriptsize]{python}
d = {'north' : 12, 'east' : 9,
     'south' :  6, 'west' : 3}

print(d['north'])      # 12
print(list(d.keys()))
# ['north', 'east', 'south', 'west']

print(list(d.items())[0:2])
# [('north', 12), ('east', 9)]

d['north'] = 0
# write access possible, too
\end{minted}
\end{codebox}
\end{columns}
%
\end{frame}

% =========================================================================== %

\begin{frame}[fragile]{Special Functions and Operators for Containers (I)}
%
\begin{itemize}
\item \inPy{in}: computes a \inPy{bool}ean: is value in container?
	\begin{itemize}
	\item \inPy{1 in [1, 2, 3]} returns \inPy{True}
	\item \inPy{[1] in [1, 2, 3]} returns \inPy{False}
	\item \inPy{1 in {1 : 'a', 2 : 'b'}} returns \inPy{True}
	\item \inPy{'a' in {1 : 'a', 2 : 'b'}} returns \inPy{False}
	\item \inPy{'a' in {1 : 'a', 2 : 'b'}.values()} returns \inPy{True}
	\end{itemize}
\item \inPy{len} -- returns number of elements within the container
\item \inPy{reversed} and \inPy{sorted} -- returns a \emph{copy} of the container with reversed or order \emph{as a \inPy{list}}
\end{itemize}
%
\end{frame}

% =========================================================================== %

\begin{frame}[fragile]{Special Functions and Operators for Containers (II)}
%
\begin{itemize}
\item Comparison operators (\texttt{==}, \texttt{<}, \texttt{>})
	\begin{itemize}
	\item Compare element-wise; decision is made at first non-equal value
	\item Cf. Strings: \inPy{"aardvark" < "antilope"} (lexicographical order)
	\item \inPy{[1, 5, 7] < [1, 5, 9, -3]} for the same reason.
	\end{itemize}
\item \inPy{zip} -- put lists \enquote{side by side}, create \texttt{tuple}s with elements from each container \\
	\vspace{-5pt}
	\begin{codebox}[example: \texttt{zip}]
	\begin{minted}[linenos, fontsize=\scriptsize]{python}
A = ["Gryffindor", "Ravenclaw", "Hufflepuff", "Slytherin"]
B = ["red", "blue", "yellow", "green"]
print(list(zip(A, B)))
	\end{minted}
	\end{codebox}
	\vspace{-10pt}
	\begin{cmdbox}[Output: \texttt{zip}]
	\begin{minted}[fontsize=\scriptsize]{text}
[("Gryffindor", "red"), ("Ravenclaw", "blue"),
 ("Hufflepuff", "yellow"), ("Slytherin", "green")]
	\end{minted}
	\end{cmdbox}
\end{itemize}
%
\end{frame}

% =========================================================================== %

\begin{frame}{Chapter 5}
%
\begin{itemize}
\item \inPy{for}-Loops
\item Unpacking of \inPy{tuple}s
\item List Comprehension
\end{itemize}
%
\end{frame}

% =========================================================================== %

\begin{frame}[fragile]{\inPy{for}-Loops}
%
\vspace{-15pt}
\begin{columns}[t]
\column{.5\linewidth}
\begin{itemize}
\item Idea: do [actions] \emph{for each element} in a container
\item Container: \inPy{list}s, \inPy{tuple}s, \inPy{range}s, \inPy{dict}s, ...
\item Action: arbitrary code (things you know already and stuff I'll show you later)
\item Can be multiple lines of code
\item[\Thus] Block structure: Indentation
\item Can be nested
\item Cf. \inPy{if}-Statements
\end{itemize}
%
\column{.5\linewidth}
\begin{codebox}[Syntax: \texttt{for}-Loops]
\begin{minted}[fontsize=\scriptsize]{python}
for element in container :
    actions

further code (independent of loop)
\end{minted}
\end{codebox}
%
\begin{itemize}
\item \texttt{element} is a \emph{new variable}
\item will be \enquote{filled} with the contents of the container, in sequence
\end{itemize}
\end{columns}
%
\end{frame}

% =========================================================================== %

\begin{frame}[fragile]
%
\begin{codebox}[Example: \texttt{for}-Loop with \texttt{list}s]
\begin{minted}[fontsize=\scriptsize, linenos]{python}
tasklist = ["write the script", "drink some coffee", "drink some more coffee"]

print("Your tasks today:")
for task in tasklist :
    print("*", task)

print("Do it now!")
\end{minted}
\end{codebox}
%
\begin{cmdbox}[Output: \texttt{for}-Loop with \texttt{list}s]
\begin{minted}[fontsize=\scriptsize]{text}
Your tasks today:
* write the script
* drink some coffee
* drink some more coffee
Do it now!
\end{minted}
\end{cmdbox}
%
\end{frame}

% =========================================================================== %

\begin{frame}[fragile]
%
\begin{codebox}[Example: \texttt{for}-loop with \texttt{range}s -- counter loops]
\begin{minted}[fontsize=\scriptsize, linenos]{python}
print("the first 10 square numbers are:")
for i in range(10) :
    print(f"{i}² = {i**2:2}")
\end{minted}
\end{codebox}
%
\begin{cmdbox}[Output: \texttt{for}-loop with \texttt{range}s -- counter loops]
\begin{minted}[fontsize=\scriptsize]{text}
the first 10 square numbers are:
0² =  0
1² =  1
2² =  4
3² =  9
4² = 16
5² = 25
6² = 36
7² = 49
8² = 64
9² = 81
\end{minted}
\end{cmdbox}
%
\end{frame}

% =========================================================================== %

\begin{frame}[fragile]
%
\begin{codebox}[Example: \texttt{for}-Loops with \texttt{tuple}s]
\begin{minted}[fontsize=\scriptsize, linenos]{python}
import math
vectors = [(1, 1), (4, 7), (-1, 2)]
for v in vectors :
    print("vector", v, "has length", math.hypot(v[0], v[1]))
\end{minted}
\end{codebox}
%
\begin{cmdbox}[Output: \texttt{for}-Loops with \texttt{tuple}s]
\begin{minted}[fontsize=\scriptsize]{text}
vector (1, 1) has length 1.4142135623730951
vector (4, 7) has length 8.06225774829855
vector (-1, 2) has length 2.23606797749979
\end{minted}
\end{cmdbox}
%
\end{frame}

% =========================================================================== %

\begin{frame}[fragile]
%
\vspace{-5pt}
\begin{codebox}[Example: \texttt{for}-Loops and unpacking of \texttt{tuple}s]
\begin{minted}[fontsize=\scriptsize, linenos]{python}
books = [("Frank Herbert", "Dune"),
         ("Douglas Adams", "The Hitchhikers Guide To The Galaxy"),
         ("Randall Munroe", "What If"),
         ("Isaac Asimov", "Foundation"),
         ("Willy Russell", "Educating Rita"),
         ("Moving Pictures", "Terry Pratchett")]
print("You should definitively read:")
for author, title in books :
    print("*", title, "by", author)
\end{minted}
\end{codebox}
%
\vspace{-10pt}
\begin{cmdbox}[Output: \texttt{for}-Loops and unpacking of \texttt{tuple}s]
\begin{minted}[fontsize=\scriptsize]{text}
You should definitively read:
* Dune by Frank Herbert
* The Hitchhikers Guide To The Galaxy by Douglas Adams
* What If by Randall Munroe
* Foundation by Isaac Asimov
* Educating Rita by Willy Russell
* Moving Pictures by Terry Pratchett
\end{minted}
\end{cmdbox}
%
\end{frame}

% =========================================================================== %

\begin{frame}[fragile]
%
\begin{hintbox}[\texttt{enumerate}]
The function \inPy{enumerate} generates \inPy{tuple}s holding the indices and elements within a container.
\end{hintbox}
%
\begin{codebox}[Example: \texttt{for}-loop with \texttt{enumerate}]
\begin{minted}[fontsize=\scriptsize, linenos]{python}
tasklist = ["write the script", "drink some coffee", "drink more coffee"]
print("Your tasks today:")
for i, task in enumerate(tasklist) :
    print(f"{i + 1}. {task}")
\end{minted}
\end{codebox}
%
\begin{cmdbox}[Output: \texttt{for}-loop with \texttt{enumerate}]
\begin{minted}[fontsize=\scriptsize]{text}
Your tasks today:
1. write the script
2. drink some coffee
3. drink more coffee
\end{minted}
\end{cmdbox}
%
\end{frame}

% =========================================================================== %

\begin{frame}[fragile]
%
\begin{codebox}[Example: \texttt{for}-Loops with \texttt{zip}]
\begin{minted}[fontsize=\scriptsize, linenos]{python}
data1 = [1.7, 2.2, -4.1]
data2 = [1.8, 2.0, -3.8]
print("Difference in datasets:")
for i, t in enumerate(zip(data1, data2)) :
    print(f"Datapoint {i}: {t} differs by : {t[0] - t[1]}")
\end{minted}
\end{codebox}
%
\begin{cmdbox}[Output: \texttt{for}-Schleifen mit \texttt{zip}]
\begin{minted}[fontsize=\scriptsize]{text}
Datapoint 0: (1.7, 1.8) differs by : -0.10000000000000009
Datapoint 1: (2.2, 2.0) differs by : 0.20000000000000018
Datapoint 2: (-4.1, -3.8) differs by : -0.2999999999999998
\end{minted}
\end{cmdbox}
%
\begin{hintbox}[Remember: \texttt{zip}]
The function \inPy{zip} generates \inPy{tuple}s from container-elements with same index.
\end{hintbox}
%
\end{frame}

% =========================================================================== %

\begin{frame}[fragile]
%
\begin{codebox}[Example: \texttt{for}-loops with \texttt{dict}s]
\begin{minted}[fontsize=\scriptsize, linenos]{python}
houseStark = {"Sigil" : "A grey direwolf on a white field",
              "Words" : "Winter Is Coming",
              "Seat"  : "Winterfell"}

print("Summary of House Stark:")
for key, value in houseStark.items() :
    print(f"{key:5}: {value}")
\end{minted}
\end{codebox}
%
\begin{cmdbox}[Output: \texttt{for}-loops with \texttt{dict}s]
\begin{minted}[fontsize=\scriptsize]{text}
Summary of House Stark:
Sigil: A grey direwolf on a white field
Words: Winter Is Coming
Seat : Winterfell
\end{minted}
\end{cmdbox}
%
\end{frame}

% =========================================================================== %

\begin{frame}[fragile]{List Comprehension}
%
Idea: Comfortably create containers from contents of another container
\begin{codebox}[Syntax: List Comprehension]
\begin{minted}[fontsize=\scriptsize]{python}
listVariable = [      expression for element in container if condition]
setVariable  = {      expression for element in container if condition}
dictVariable = {key : expression for element in container if condition}
\end{minted}
\end{codebox}
%
\begin{itemize}
\item \texttt{expression} : arbitrary \enquote{computation}, describing an element of the new container \texttt{XXXVariable}
\item \texttt{element}    : helper variable, receiving elements from \texttt{container} (as seen with \inPy{for})
\item \texttt{container}  : A \inPy{list}, \inPy{range}, ...
\item \texttt{condition}  : expression that tells which elements to actually include in the new container
\item The \inPy{if condition} clause is optional
\end{itemize}
%
\end{frame}

% =========================================================================== %

\begin{frame}[fragile]
%
\begin{codebox}[Example: List Comprehension]
\begin{minted}[fontsize=\scriptsize, linenos]{python}
import math

N = 3
eValues = [math.exp(k) for k in range(N)]

print(eValues)
\end{minted}
\end{codebox}
%
\begin{cmdbox}[Output: List Comprehension]
\begin{minted}[fontsize=\scriptsize]{text}
[1.0, 2.718281828459045, 7.38905609893065]
\end{minted}
\end{cmdbox}
%
\end{frame}

% =========================================================================== %

\begin{frame}[fragile]
%
\begin{codebox}[Example: Telephone-Matrix with List Comprehension]
\begin{minted}[fontsize=\scriptsize, linenos]{python}
telephone = [[3 * row + column + 1 for column in range(3)]
             for row in range(3)
            ]

for line in telephone :
    print(line)

print(telephone)
\end{minted}
\end{codebox}
%
\begin{cmdbox}[Output: Telephone-Matrix with List Comprehension]
\begin{minted}[fontsize=\scriptsize]{text}
[1, 2, 3]
[4, 5, 6]
[7, 8, 9]
[[1, 2, 3], [4, 5, 6], [7, 8, 9]]
\end{minted}
\end{cmdbox}
%
\end{frame}

% =========================================================================== %

\begin{frame}[fragile]
%
\begin{codebox}[Example: Prime numbers with List Comprehension]
\begin{minted}[fontsize=\scriptsize, linenos]{python}
N = 100

primeNumbers = [
    i for i in range (2, N)          # take all numbers i between 2 and N ...
    if len (                         # ... for which the number of divisors ...
        [j for j in range (1, i+1)   # ... (i.e. numbers between 1 und i ...
        if i % j == 0]               # ... that give no remainder) ...
    ) == 2                           # ... is equal to 2.
]

print(primeNumbers)

\end{minted}
\end{codebox}
%
\begin{cmdbox}[Output: Prime numbers with List Comprehension]
\begin{minted}[fontsize=\scriptsize]{text}
[2, 3, 5, 7, 11, 13, 17, 19, 23, 29, 31, 37, 41, 43, 47, 53, 59, 61, 67, 71,
 73, 79, 83, 89, 97]
\end{minted}
\end{cmdbox}
%
\end{frame}
